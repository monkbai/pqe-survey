%%%%%%%%%%%%%%%%%%%%%%%%%%%%%%%%%%%%%%%%%%%%%%%%%%%%%%%%%%%%%%%%%%%%%%%%%
%                                                                       %
% ustthesis_test.tex: A template file for usage with ustthesis.cls      %
%                                                                       %
%%%%%%%%%%%%%%%%%%%%%%%%%%%%%%%%%%%%%%%%%%%%%%%%%%%%%%%%%%%%%%%%%%%%%%%%%

\documentclass{ustthesis}

\usepackage{mathpazo,amsmath,amssymb,epsfig,enumerate,bbm,calc,color,ifthen,capt-of} % original was times, but I think it's ugly; we use the same as IEEE CompSoc
\usepackage{algorithm}
\usepackage[noend]{algorithmic}
\usepackage[center]{subfigure}
\usepackage{color,graphicx}
\newtheorem{proof}{Proof}
\usepackage{hyperref} % for better viewing experience  -- added by alan
\usepackage[margin=25mm,textheight=247mm,textwidth=145mm]{geometry}

% Alan: begin the font trial
% Euler for math | Palatino for rm | Helvetica for ss | Courier for tt
\renewcommand{\rmdefault}{ppl} % rm
%\linespread{1.05}        % Palatino needs more leading
\usepackage[scaled]{helvet} % ss
\usepackage{courier} % tt
%\usepackage{euler} % math
\usepackage{eulervm} % a better implementation of the euler package (not in gwTeX)
\normalfont
\usepackage[T1]{fontenc}
% Alan: end the font trial

\newcommand{\red}[1]{#1}
\newcommand{\tab}[1]{\hspace{3mm}}

% \usepackage{latexsym}
    % Use the "latexsym" package when encountering the following error:
    %   ! LaTeX Error: Command \??? not provided in base LaTeX2e.
% \usepackage{epsf}
    % Use the "epsf" package for including EPS files.

%%%%%%%%%%%%%%%%%%%%%%%%%%%%%%%%%%%%%%%%%%%%%%%%%%%%%%%%%%%%%%%%%%%%%%%%%
%                                                                       %
% Preambles. DO NOT ERASE THEM. Change to suite your particular purpose.%
%                                                                       %
%%%%%%%%%%%%%%%%%%%%%%%%%%%%%%%%%%%%%%%%%%%%%%%%%%%%%%%%%%%%%%%%%%%%%%%%%

\title{A Survey on Visualization for Explainable Classifiers}  % Title of the thesis.
\author{Yao~MING}     % Author of the thesis.
% \degree{\MPhil}             % Degree for which the thesis is.
%% or
%\degree{\PhD}              % Degree for which the thesis is.
\subject{Computer Science and Engineering}      % Subject of the Degree.
\department{Computer Science and Engineering}       % Department to which the thesis
                    % is submitted.
\advisor{Prof.~Huamin~Qu}     % Supervisor.
% \depthead{Prof.~Mounir~HAMDI}    % department head.
% \defencedate{2010}{06}{25}      % \defencedate{year}{month}{day}.

% NOTE:
%   According to the sample shown in the guidelines, page number is
%   placed below the bottom margin.  However, if the author prefers
%   the page number to be printed above the bottom margin, please
%   activate the following command.

% \PNumberAboveBottomMargin

\begin{document}

%%%%%%%%%%%%%%%%%%%%%%%%%%%%%%%%%%%%%%%%%%%%%%%%%%%%%%%%%%%%%%%%%%%%%%%%%
%                                                                       %
% Now the actual Thesis. The order of output MUST be followed:          %
%                                                                       %
%    1) TITLEPAGE                                                       %
%                                                                       %
% The \maketitle command generates the Title page as well as the        %
% Signature page.                                                       %
%                                                                       %
%%%%%%%%%%%%%%%%%%%%%%%%%%%%%%%%%%%%%%%%%%%%%%%%%%%%%%%%%%%%%%%%%%%%%%%%%

\maketitle

%%%%%%%%%%%%%%%%%%%%%%%%%%%%%%%%%%%%%%%%%%%%%%%%%%%%%%%%%%%%%%%%%%%%%%%%%
%                                                                       %
%     2) DEDICATION (Optional)                                          %
%                                                                       %
% The \dedication and \enddedication commands are optional. If          %
% specified it generates a page for dedication.                         %
%
%%%%%%%%%%%%%%%%%%%%%%%%%%%%%%%%%%%%%%%%%%%%%%%%%%%%%%%%%%%%%%%%%%%%%%%%%

% \dedication
% This is an optional section.
% \enddedication

%%%%%%%%%%%%%%%%%%%%%%%%%%%%%%%%%%%%%%%%%%%%%%%%%%%%%%%%%%%%%%%%%%%%%%%%%
%                                                                       %
%     3) ACKNOWLEDGMENTS                                                %
%                                                                       %
% \acknowledgments and \endacknowledgments defines the                  %
% Acknowledgments of the author of the Thesis.                          %
%                                                                       %
%%%%%%%%%%%%%%%%%%%%%%%%%%%%%%%%%%%%%%%%%%%%%%%%%%%%%%%%%%%%%%%%%%%%%%%%%

% \input{chapter/sec-ack}
%%%%%%%%%%%%%%%%%%%%%%%%%%%%%%%%%%%%%%%%%%%%%%%%%%%%%%%%%%%%%%%%%%%%%%%%%
%                                                                       %
%     4) TABLE OF CONTENTS                                              %
%                                                                       %
%%%%%%%%%%%%%%%%%%%%%%%%%%%%%%%%%%%%%%%%%%%%%%%%%%%%%%%%%%%%%%%%%%%%%%%%%

\tableofcontents

%%%%%%%%%%%%%%%%%%%%%%%%%%%%%%%%%%%%%%%%%%%%%%%%%%%%%%%%%%%%%%%%%%%%%%%%%
%                                                                       %
%     5) LIST OF FIGURES (If Any)                                       %
%                                                                       %
%%%%%%%%%%%%%%%%%%%%%%%%%%%%%%%%%%%%%%%%%%%%%%%%%%%%%%%%%%%%%%%%%%%%%%%%%

% \listoffigures

%%%%%%%%%%%%%%%%%%%%%%%%%%%%%%%%%%%%%%%%%%%%%%%%%%%%%%%%%%%%%%%%%%%%%%%%%
%                                                                       %
%     6) LIST OF TABLES (If Any)
%                                                                       %
%%%%%%%%%%%%%%%%%%%%%%%%%%%%%%%%%%%%%%%%%%%%%%%%%%%%%%%%%%%%%%%%%%%%%%%%%

% \listoftables

%%%%%%%%%%%%%%%%%%%%%%%%%%%%%%%%%%%%%%%%%%%%%%%%%%%%%%%%%%%%%%%%%%%%%%%%%
%                                                                       %
%     7) ABSTRACT                                                       %
%                                                                       %
% \abstract and \endabstract are used to define a short Abstract for    %
% the Thesis.                                                           %
%                                                                       %
%%%%%%%%%%%%%%%%%%%%%%%%%%%%%%%%%%%%%%%%%%%%%%%%%%%%%%%%%%%%%%%%%%%%%%%%%

\begin{abstract}

Classification is a fundamental problem in machine learning, data mining and computer vision. In practice, interpretability is a desired property of classification models (classifiers) in critical areas like security, medicine and finance. For instance, a quantitative trader may prefer a more interpretable model with less expected return due to its predictability. Unfortunately, most best-performing classifiers in many applications (e.g., deep neural networks) are complex machines whose predictions are difficult to explain. Thus, there is a growing interest in using visualization to understand, diagnose and explain machine learning systems in both academia and industry. Many challenges need to be addressed in the formalization of explainability and design principles and evaluation of explainable intelligent systems.

The survey starts with an introduction on the concept and background of explainable classifiers. Existing work in both visualization and machine learning communities is categorized in terms of data types and purposes of explanation. Then the survey ends with a discussion on the challenges and future research opportunities of explainable classifiers.

\end{abstract}


%%%%%%%%%%%%%%%%%%%%%%%%%%%%%%%%%%%%%%%%%%%%%%%%%%%%%%%%%%%%%%%%%%%%%%%%%
%                                                                       %
%     8) The Actual Contents                                            %
%                                                                       %
% The command \chapters MUST BE USED to ensure that the entire content  %
% of the Thesis is double-spaced (in version 1.0).                      %
%                                                                       %
% However, in version 2.0, \chapters will be automatically added in     %
% the beginning of the first chapter.                                   %
%                                                                       %
%%%%%%%%%%%%%%%%%%%%%%%%%%%%%%%%%%%%%%%%%%%%%%%%%%%%%%%%%%%%%%%%%%%%%%%%%

%%\chapters         % Not necessary with ustthesis.cls (v2.0).

%%%%%%%%%%%%%%%%%%%%%%%%%%%%%%%%%%%%%%%%%%%%%%%%%%%%%%%%%%%%%%%%%%%%%%%%%
%                                                                       %
% Each chapter is defined via the \chapter command. The usual sectional %
% commands of LaTeX are also available.                                 %
%                                                                       %
%%%%%%%%%%%%%%%%%%%%%%%%%%%%%%%%%%%%%%%%%%%%%%%%%%%%%%%%%%%%%%%%%%%%%%%%%


\chapter{Introduction}\label{sec-introduction}

Placeholder for introduction.

\newpage

\chapter{Conclusion}\label{sec-conclusion}

Explainability is a critical but often-overlooked property for intelligent systems. In this survey, we review techniques that provide explainability to classifiers, with a special focus on visualization. We first discuss the concept and definition of explainability of a classifier. Two types of techniques that provide explainability to classifiers are summarized and discussed. Then based on the life cycle of intelligent systems, we discuss the role of visualization in different stages, and how visualization can be used to improve explainability.

The research in the explainability of classifiers is still a growing discipline. There is no consensus on the definition of explainability in the context of supervised learning. There is few rigorous evaluation methods of the explainability of a classifier or a explanation of a classifier. Early work treats the explainability as the ``simplicity'' and always focuses on balancing the trade-offs between performance and simplicity. Now there is a new trend of building an explanatory interface between human and the underlying model to enhance explainability while maintaining the performance. A few challenges and research issues are summarized as follows.

\textbf{Rigorous theory of explainability}. There are plenty of unsolved questions in the the seemly intuitive concept of explainability. How to rigorously define explainability in the context of machine learning or artificial intelligence? What is explanation? How to evaluate whether an explanation is good or not? How can we model the variation and uncertainty of human in the explanatory interface? Based on the theory of explainability, a further issue that needs to address is the \textbf{evaluation} of explainability and the quality of explanations. If a metric based evaluation is inapplicable, there still need guidelines for system designs. There is some work in cognitive science studying the function, structure of explanation, and the role that explanation plays in perception, cognition, and learning. It may be promising to learn from the theory from cognition science. Still, it requires the efforts from both cognitive science and computer science.

\textbf{Applications}. Theory comes from practice. Although there are toy examples, showing how explainability helps design better models, avoid possible bias, and enhance users' trust, we still lack knowledge on the design challenges of a explainable interface in real-world applications. Is it possible to let the machine explain its learned knowledge to a human? There are few successes or failures in real-world applications that we can learn from. Another neglected stakeholder is the end-users who actually use or are influenced by the technology developed based on AI. How do they value explainability during the use of an intelligent system? How can we use explanatory techniques to improve their using experience? The research at this end might be able to impact more people.

% \textbf{}


% \section{Other Applications}
% \subsection{Teaching and Communicating Models}
% \cite{harley2015isvc}
% Basilio Noris develop a visualization tool, MLDemos\footnote{\url{http://mldemos.epfl.ch/}}, for understanding how different algorithms and parameters influence the results in different machine learning problems. The tool project the decision space back to the input space,  how the output space of a classifier is 
% \cite{wongsuphasawat2017dataflow}
% Narrative, Interactive, etc. to explain your model to others.
% \subsection{Learn from the Model} 
% Knowledge Discovery; Learn lessons from what the model learned (Alpha Go)

%%%%%%%%%%%%%%%%%%%%%%%%%%%%%%%%%%%%%%%%%%%%%%%%%%%%%%%%%%%%%%%%%%%%%%%%%
%                                                                       %
%      9) BIBLIOGRAPHY                                                  %
%                                                                       %
% This example uses bibtex to generate the required Bibliography. Refer %
% to the % the file ustthesis_test.bib for the entries of the           %
% Bibliography. Note that only the cited entries are printed.           %
%                                                                       %
% If BibTeX is not used to typeset the bibliography, replace the        %
% following line with the \begin{thebibliography} and \end{bibliography}%
% commands (the "thebibliography" environment) to process the           %
% Bibliography.                                                         %
%                                                                       %
%%%%%%%%%%%%%%%%%%%%%%%%%%%%%%%%%%%%%%%%%%%%%%%%%%%%%%%%%%%%%%%%%%%%%%%%%

%%%%%%%%%%%%%%%%%%%%%%%%%%%%%%%%%%%%%%%%%%%%%%%%%%%%%%%%%%%%%%%%%%%%%%%%%
%                                                                       %
% The recommended bibliography style is the IEEE bibliography style.    %
% "ustbib" defines the IEEE bibliography standard with the added        %
% ability of sorting the items by name of author.                       %
%                                                                       %
% If you are not using BibTeX to process your Bibliography, comment out %
% the following line.                                                   %
%                                                                       %
%%%%%%%%%%%%%%%%%%%%%%%%%%%%%%%%%%%%%%%%%%%%%%%%%%%%%%%%%%%%%%%%%%%%%%%%%

\bibliographystyle{plain}

\bibliography{ref}
% Please run "bibtex ustthesis_test" before the bibliography can be
% included.

%%%%%%%%%%%%%%%%%%%%%%%%%%%%%%%%%%%%%%%%%%%%%%%%%%%%%%%%%%%%%%%%%%%%%%%%%
%                                                                       %
%     10) APPENDIX (If Any)                                              %
%                                                                       %
% \appendix command marks the beginning of the APPENDIX part of the     %
% Thesis. The usual \chapter command is used for the different chapters %
% of the Appendix.                                                      %
%                                                                       %
%%%%%%%%%%%%%%%%%%%%%%%%%%%%%%%%%%%%%%%%%%%%%%%%%%%%%%%%%%%%%%%%%%%%%%%%%


%%%%%%%%%%%%%%%%%%%%%%%%%%%%%%%%%%%%%%%%%%%%%%%%%%%%%%%%%%%%%%%%%%%%%%%%%
%                                                                       %
%     11) BIOGRAPHY (Optional)                                          %
%                                                                       %
% \biography and \endbiography are used to define the optional          %
% Biography of the author of the Thesis.                                %
%                                                                       %
%%%%%%%%%%%%%%%%%%%%%%%%%%%%%%%%%%%%%%%%%%%%%%%%%%%%%%%%%%%%%%%%%%%%%%%%%

% \biography
% The biography of the student is ALSO optional.
% \endbiography

\end{document}
