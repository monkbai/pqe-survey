\section{Lifting} \label{sec:existing-lifting}
While disassembly code faithfully expresses the semantics of the program, it 
operates on the low-level machine architecture, such as registers and memory, 
and therefore contains much cumbersome information that is unnecessary for 
understanding the program. Disassembly code is usually refined and lifted into 
higher-level IR for advanced binary analysis and further understanding of the 
program. In this section, we introduce and discuss existing IR lifter solutions 
and related reverse engineering techniques.

\subsection{LLVM IR Lifting} \label{sec:existing-llvm-lifting}
% SecondWrite
\noindent\textbf{Static Lifter.}~SecondWrite is the first reverse engineering 
tool that lifted disassembly code into a compiler-level IR (LLVM IR) to reuse 
complex high-level transformations provided by LLVM framework~\cite
{anand2013compiler,elwazeer2013scalable}.
SecondWrite proposed simple stack frame analysis to deconstruct physical stack 
into individual abstract stack frames and an algorithm to promote memory 
locations to symbols aided with Value Set Analysis (VSA)
~\cite{balakrishnan2004analyzing}.
They also demonstrate the necessity of abstract and symbols for improved 
dataflow analysis and readability.
However, this method does not rely on any debug or symbol information and 
therefore faces the same challenges mention in Chapter \ref{sec-challenges}. 
While this approach performs well on their data sets, as compilers have changed 
and evolved in recent years, it will likely fail on non-trivial binaries.

% BinRec
\noindent\textbf{Dynamic Lifter.}~A well-developed LLMV IR Lifter would be of 
great benefit to the whole reverse engineering community. However, although the 
IR lifter has great application prospects, exploration of the IR lifting 
technique is still limited due to many difficulties, including variables 
recovery, types recovery, and translation from assembly code to IR.
These difficulties mean that developing a lifter requires a lot of human 
resources and time. There has been no new lifter technique in academia for a 
long time after SecondWrite. Until recently, a new dynamic IR lifter named 
BinRec was proposed~\cite{altinay2020binrec}.
To avoid the trap of variables and types recovery, BinRec took a different 
direction by recording the execution trace and later simulating the program 
execution from the hardware level.
Particularly, BinRec is built on top of S$^2$E~\cite{chipounov2011s2e}, a 
symbolic execution framework running in the QEMU virtual machine
~\cite{bellard2005qemu}.
Given a binary program to be lifted, S2E is first used to collect as many 
distinct execution traces as possible. These traces will be merged and 
translated into one LLVM IR file, which will stitch with the original program 
into a new rewritten binary program.

However, this dynamic lifter has two primary defects that significantly reduce 
the range of its application. First, BinRec employs the 
\textit{emulation-style} translation strategy that uses LLVM IR to simulate 
hardware-level program execution. Without recovered higher-level information, 
such as variables, types, functions, almost all transformations provided by the 
LLVM framework would not be applied on such IR. Second, the lifted program can 
only run properly on covered paths, while improving coverage is another 
frequently studied problem.

\noindent\textbf{Industrial Lifters.}~Except for academic tools discussed 
above, some open-source lifters developed by technical companies also achieved 
great success. For example, McSema, another emulation style lifter developed by 
Trail of Bits~\cite{trailofbits}, is considered one of the best lifters 
available today~\cite{mcsema}.
McSema is also using IR statements to simulate assembly instructions from the 
hardware level. Specifically, it uses global variables to represent the machine 
architecture, such as registers and memory, and uses IR statements to read and 
write these global variables, just like assembly instructions read and write 
machine architecture.

\begin{figure}[tb]
  \centering
  \includegraphics[width=1.0\textwidth]{fig/mcsema.pdf}
  \caption{A case study comparing LLVM IR lifted by McSema with LLVM IR compiled by Clang.}
  \label{fig:mcsema}
\end{figure}

As illustrated in \F~\ref{fig:mcsema}, IR lfited by McSema also does not 
contain variables, types, function signatures, and other higher-level 
information, which makes McSema lifted IR not a suitable target for 
transformation passes provided by the LLVM framework. The difference with 
BinRec is that McSema leverage the cutting-edge commercial reverse engineering 
tool, IDA~\cite{hex2014ida}, for static CFG recovery. Therefore McSema can 
statically transform and optimize the entire program without worrying about 
coverage.

On the other hand, Avast~\cite{avast} developed RetDec~\cite{retdec}, as a 
succinct-style lifter, tries to lift assembly to IR close to compiler-generated 
IR. However, as we discussed in \S~\ref{sec:challenges-variable}, recovering 
symbol or debugging information is unsolved. In most cases, RetDec cannot lift 
non-trivial programs into IR correctly and results in unfunctional rewritten 
binaries. Therefore, the output of RetDec is only used for tasks that do not 
require precise semantics, such as machine-learning-based program 
classification~\cite{ben2018neural} and manual inspection.


\subsection{Variables and Types Recovery} \label{sec:existing-recovery}
Although RetDec can hardly produce functional IR for non-trivial programs, it 
shows great potential in the program classification task. As shown in 
\F~\ref{fig:ncc}, we apply ncc (Neural Code Comprehension) method
~\cite{ben2018neural} on five different lifter settings and compare the 
classification results with ncc based on compiler-generated IR. Mctoll
~\cite{mctoll} is another \textit{succinct-style} lifter similar to RetDec, and 
McSema- represents McSema with all optimizations disabled. Lifters like RetDec 
and Mctoll show encouraging support for the classification task, whose 
performance is comparable to compiled IR.
Also, we find that lifters like RetDec have only implemented rudimentary 
analysis passes instead of advanced research products in the variables and 
types recovery field. So in this section, we will discuss recent variables 
recovery techniques that may further raise the upper bound of lifters.

\begin{figure}[tb]
  \centering
  \includegraphics[width=1.0\textwidth]{fig/ncc.pdf}
  \caption{Classification analysis results of each tool across all 104 classes of the POJ-104 dataset.}
  \label{fig:ncc}
\end{figure}

\noindent\textbf{Dynamic Approaches.}~
% REWARDS
REWARDS uses a dynamic ``type sink'' technique to reveal program data 
structures from binaries automatically~\cite{lin2010automatic}. To be more 
specific, a timestamped attribute will be assigned to each memory location 
accessed. This attribute is propagated to other memory locations and registers 
during program execution. Once a type-revealing point or ``type sink'' that can 
resolve a variable's type is executed (e.g., the type revealing instructions, 
system calls, and library calls), the types of all memory locations sharing the 
same attribute are resolved. Similarly, another dynamic solution, Howard, aims 
to recover data structures from memory access patterns in execution traces
~\cite{slowinska2011howard}. These dynamic approaches, while effective, are 
limited by coverage issues.

\noindent\textbf{Static Approaches.}~
% TIE
TIE proposes an end-to-end type inference system~\cite{lee2011tie}. It starts 
by lifting binary to BIL (BAP Intermediate Language) and finding variables with 
a variation of the VSA method named DVSA. The type system that consists of type 
inference rules is then used to generate constraints based on the BIL. Finally, 
the constraints are solved and lead to types as the results. TIE can produce 
accurate and conservative results, however, the heavyweight data flow analysis 
used for variable recovery makes it less practical in large binaries.
% SecondWrite
In the same way, SecondWrite~\cite{elwazeer2013scalable} makes a trade-off 
between accuracy and speed by combining a best-effort VSA variant for points-to 
analysis with a type-inference engine. Note that accurate types depend on 
high-quality points-to data, therefore less accurate points-to analysis will 
lead to a decrease in the accuracy of the results. Also, even original VSA is 
known to produce a lot of inaccurate results.
% Retypd
Comparably, Retypd (\textbf{re}gular \textbf{ty}pes using 
\textbf{p}ush\textbf{d}own) built on top of GrammaTech’s machine-code-
analysis tool CodeSurfer~\cite{balakrishnan2005codesurfer} introduces a 
principled static type-inference algorithm that supports more advanced types, 
including recursive types, polymorphism, and subtyping
~\cite{noonan2016polymorphic}.

% Osprey
One of the latest probabilistic variables recovery techniques, \textsc{Osprey} 
(rec\underline{O}very of variable and data \underline{S}tructure by 
\underline{PR}obabilistic analysis for stripp\underline{E}d binar\underline{Y})
~\cite{zhang2021osprey}, leverages a more advanced Path Sampling Driven 
Per-path Abstract Interpretation named BDA~\cite{zhang2019bda} instead of VSA 
for accurate and efficient points-to analysis. Moreover, as inevitable 
uncertainty in variable and structure recovery may result in contradicting 
predictions, \textsc{Osprey} introduces random variables in the type inference 
system to model the probabilities that a memory location is of a specific type. 
During type inference, probabilistic constraints are derived from these random 
variables and will be further solved to produce posterior probabilities as the 
recovery results. \F~\ref{fig:osprey} shows that \textsc{Osprey} achieves 
around 90\% precision, which substantially outperforms state-of-the-art 
commercial and open-source tools, including IDA~\cite{hex2014ida}, Ghidra
~\cite{ghidra}, Angr~\cite{shoshitaishvili2016sok}, and Howard
~\cite{slowinska2011howard}.

\begin{figure}[tb]
  \centering
  \includegraphics[width=1.0\textwidth]{fig/OSPREY.pdf}
  \caption{Presision and Recall of \textsc{Osprey}~\cite{zhang2021osprey}.}
  \label{fig:osprey}
\end{figure}

\noindent\textbf{ML-bassed Approaches.}~
% Debin
Machine learning has also been applied to recover variables and types. \textsc
{Debin}~\cite{he2018debin} presents the first machine-learning-based approach 
to predict debugging information in stripped binaries. It combines an Extremely 
randomized Tree (ET)~\cite{geurts2006extremely} classification model with a 
linear probabilistic graphical model to recover properties of variables (e.g., 
symbol names, types, and locations). ET is used to extract unknown and known 
elements (variables) from lifted BAP-IR, then the linear probabilistic 
graphical model makes joint predictions on the debug information of elements. 
This approach effectively predicts symbol names and types with 68.8\% precision 
and 68.3\% recall on the x64 platform.
% CATI
CATI (Context-Assisted Type Inference)~\cite{chen2020cati} defines a new 
feature called Variable Usage Context (VUC) for type inference. The basic idea 
is based on the observation that neighboring instructions are likely to operate 
the same type of variables. Also, according to the empirical survey, only about 
65\% of variables have more than 3 related instructions, which means 35\% of 
variables have only 1 or 2 related instructions. Thus to further improve the 
type inference accuracy, CATI first extracts the VUC feature that contains the 
target instruction with instruction context, then feeds the VUC feature into 
Word2Vec~\cite{mikolov2013distributed} model for assembly code embedding, 
finally, a multi-stage classifier with a convolutional neural network (CNN) is 
used to predict type for the target variable. The evaluation shows that CATI 
can achieve 71.2\% accuracy for type inference on unseen stripped binaries.


\subsection{Lifter Verification} \label{sec:existing-lifter-verfication}
Writing a precise binary lifter is extremely difficult even for those heavily 
tested projects, thus the verification of Lifter is an indispensable part of 
lifter-related research.
% MeanDiff ASE 2017
MeanDiff~\cite{kim2017testing} is the first lifter verifier that formally 
checks the semantic equivalence. It tested 3 binary lifters, including BAP~\cite
{brumley2011bap}, PyVEX~\cite{pyvex}, and BINSEC~\cite{bardin2011bincoa}, by 
translating binary-based IRs (BBIRs) to a Unified Intermediate Representation 
(UIR) and generating symbolic summaries with data-flow analysis and symbolic 
execution. Although this method works on some low-level (does not contain 
variables information) IRs, it cannot be used to test LLVM IR lifters because 
it is more challenging to define a unified representation for lifted LLVM IR, 
in the sense that lifters aiming at different applications produce distinct 
results (i.e., emulation-style and succinct-style).

% PLDI 2020
Dasgupta et al.~\cite{dasgupta2020scalable} proposed a new lifter verification 
framework based on their previous work, in which they provided a complete 
formal semantics of x86-64 user-level instruction set architecture~\cite
{dasgupta2019complete}. Their work is the first single-instruction translation 
validation framework using the formal semantics of x86-64 and IR. Similar to 
MeanDiff, it also uses symbolic execution to extract symbolic summaries from 
x86-64 and IR. With this framework, they proved that McSema is able to 
correctly translate 2254/2348 functions taken from LLVM's single-source 
benchmark test suite.
