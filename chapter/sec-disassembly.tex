\section{Disassembly} \label{sec:existing-disassembly}
As discussed in Chapter \ref{sec-challenges}, one of the major difficulties in
disassembly is differentiating code and data. While some successful disassembly
tools exist on the market~\cite{hex2014ida,kvroustek2017retdec,ghidra,radare},
these tools inevitably have substantial false positives (take data as code) and
false negatives (ignore code as data). In this section, we mainly discuss
several state-of-the-art disassembly and binary rewriting techniques that
solved the problem to a large extent.

\subsection{Disassembly with Less Errors} \label{sec:existing-less-errors}
Bauman et al. proposed \textit{superset disassembly} that employs the brute 
force method to guarantee no false negatives~\cite{bauman2018superset}. 
Specifically, superset disassembly takes every offset in the text segment as 
one possible start of an instruction, called superset instruction. It finds 
intended sequences of instructions by brute force disassembling every possible 
instruction, although the bytes of adjacent instructions may overlap.
%
In this way, it can be guaranteed that there is no false positive in the 
result, and thus enabling error-free binary rewriting. However, the 
disassembled program will be bloated as lots of false-positive instructions 
exist. The bloated instructions could cause substantial size and runtime 
overhead because getting the instruction to be executed needs a table lookup 
each time, especially in practice, a binary writer based on superset 
disassembly has to instrument all superset instructions.
% probabilistic disassembly
Thus, Miller et al. proposed \textit{probabilistic disassembly} based on 
superset disassembly to reduce the number of false positives further
~\cite{miller2019probabilistic}. Probabilistic disassembly aims to reason the 
inherent uncertainty in the binary analysis caused by the lack of debugging 
information. The basic idea is to compute the probabilities of each address 
being the true positive start of an instruction. They define three hints that 
imply true positive instructions and assign prior probabilities to these hints, 
then perform probabilistic inference to summarize the confidence of true 
positives from these evidence.

% Datalog disassembly
Besides the false negatives free but costly disassembly approaches, some 
research aims to disassembly faster while keeping a low false positives/
negatives rate. Flores-Montoya \etal present \texttt{Ddisasm}~\cite
{flores2020datalog}, a disassembly framework based on Datalog combining static 
analysis and heuristics, achieving lower false positives and false negatives 
rates compared with \texttt{Ramblr}~\cite{wang2017ramblr}. This framework takes 
advantage of Datalog, enabling faster empirical evaluation of new heuristics 
and analyses.
% XDA NDSS 2021 Suman Jana
In addition, \texttt{XDA} (Xfer-learning DisAssembler) leverages transfer 
learning to use different contextual dependencies learned from machine code for 
accurate disassembly~\cite{pei2020xda}. Although this framework surpasses SOTA 
disassemblers on both speed and accuracy, errors in results cannot be easily 
fixed without retraining, which prevents machine learning-based frameworks from 
being iteratively updated frequently. Also, it requires expensive GPUs to 
outpace existing tools on speed.

\subsection{Binary Rewriting} \label{sec:existing-symbolization}
% Uroboros
Even though we can get correct disassembly code, there is still a huge gap 
between achieving successful binary rewriting. This gap is summarized as the 
symbolization problem or relocatable problem~\cite{wang2015reassembleable,
wang2017ramblr}, as discussed in \S~\ref{sec:challenges-symbol}. \texttt
{Uroboros} was the first disassembler to be capable of not only disassembling 
code, but also symbol information from striped binaries automatically~\cite
{wang2015reassembleable}. \F~\ref{fig:uroboros} illustrates c2c(code to code), 
c2d(code to data), d2d(data to data), and d2c(data to code), in total four 
types of symbol references defined in \texttt{Uroboros}. To solve the 
symbolization problem, \texttt{Uroboros} summarized several heuristics to 
distinguish these four types and thus successfully rewrote all \textsc
{Coreutils}, \textsc{SPEC INT 2006}, and 7 real-world programs. More 
importantly, after correctly recovering symbol information, \texttt{Uroboros} 
is able to disassemble-reassemble iteration 1,000 times, in which the output of 
each iteration becomes the input of the next round, with almost zero code size 
explosion and speed slowdown.

\begin{figure}[tb]
  \centering
  \includegraphics[width=0.6\textwidth]{fig/uroboros.pdf}
  \caption{Different types of symbol references in assembly code~\cite{wang2015reassembleable}.}
  \label{fig:uroboros}
\end{figure}

% Ramblr
However, manually designed heuristics can hardly achieve binary rewriting 
across compiler versions because of various optimization and code generation 
strategies.
\texttt{Ramblr} proposed more sophisticated heuristics and analyses for 
symbolization. Based on the \texttt{angr} binary analysis platform~\cite
{shoshitaishvili2016sok}, \texttt{Ramblr} is able to rewrite all \textsc
{Coreutils} programs and 143 binaries from the DARPA Cyber Grand Challenge 
(CGC) on newer compilers (\texttt{gcc} 5.4.1 and \texttt{Clang} 4.4) with more 
optimization levels (\texttt{O0}, \texttt{O1}, \texttt{O2}, \texttt{O3}, \texttt
{Ofast}, and \texttt{Os}). Although Ramblr went further than the most advanced 
binary rewriting technique of the time, it did not go beyond the inherent 
limitations of heuristics and was still error-prone on higher versions of 
compilers.

\subsection{PIC Binary Rewriting} \label{sec:existing-pic}
Generally speaking, recovering symbol information from stripped binaries has to 
rely on complicated binary analysis or error-prone heuristics, and thus 
efficient and error-free static binary rewriting is difficult. With the trend 
of PIC (Position-Independent Code) becoming the default option of mainstream 
compilers, some of the subsequent research turns to narrow down their research 
scopes to 64-bit PIC code only to circumvent the challenging symbolization 
problem.
% RetroWrite
\texttt{RetroWrite} designed a principled symbolization strategy without 
heuristics~\cite{dinesh2020retrowrite}. \texttt{RetroWrite} leverages 
relocation information existing in PIC binaries to identify symbolizable 
constants. More specifically, it recovers symbols by
\begin{itemize}
  \item[1)] converting targets of control-flow instructions (i.e., calls and jumps) to symbols (recovering code-to-code references),
  \item[2)] converting the PC-relative addresses to symbols (recovering code-to-code and code-to-data references) to further complete the control flow graph, and
  \item[3)] converting data relocations to symbols (recovering data-to-data and data-to-code references).
\end{itemize}
In this way, \texttt{RetroWrite} presented a sound and scalable symbolization 
approach, which makes it robust enough to support multiple security-critical 
downstream applications, such as \texttt{AddressSanitizer}~\cite
{serebryany2012addresssanitizer} and coverage-guided greybox fuzzing~\cite
{zalewski2014american}.

% Egalito
Similarly, \texttt{Egalito} uses the relocation information that exists in 
x86-64 and ARM64 binaries to tackle the symbolization problem. Egalito first 
lifts disassembled code into the machine-specific, layout-agnostic Egalito IR, 
then identifies code pointers and reconstructs jump tables on the IR, thus 
theoretically enabling cross-platform binary rewriting. Their evaluation shows 
that \texttt{Egalito} is adequate to augment hardware/compiler deployment with 
multiple defense techniques, including a retpoline defense against Spectre 
~\cite{kocher2019spectre}, a software implementation of Intel’s CET~\cite{cet}, 
and a continuous code randomization defense named JIT-Shuffling~\cite
{williams2016shuffler}.

\subsection{Others} \label{sec:existing-dis-others}

\begin{figure}[tb]
  \centering
  \includegraphics[width=0.6\textwidth]{fig/superset.pdf}
  \caption{Overview of \textsc{Multiverse}~\cite{bauman2018superset}.}
  \label{fig:superset}
\end{figure}

% Superset / Probabilistic disassembly way
Except for recovering symbol information precisely, another line of research 
tries to sacrifice code size and performance for stable binary rewriting. For 
example, \textsc{Multiverse}~\cite{bauman2018superset,miller2019probabilistic} 
appends the disassembled code and a mapping from old addresses to new addresses 
after the original binary as the \texttt{.newtext} segment and \texttt{.
localmapping} segment, as illustrated in \F~\ref{fig:superset}. The original 
\texttt{.text} segment is kept as read-only data so that recompiled elf binary 
could read function pointers (code-to-code and data-to-code references) from it 
and translate the pointers to its new address by looking up the \texttt{.
localmapping} segment.

% E9Patch
On the other hand, \textsc{E9Patch} introduced a new idea to achieve binary 
rewriting without control flow recovery~\cite{duck2020binary}. For the purpose 
of more flexible static x86-64 binary rewriting, \textsc{E9Patch} came up with 
a set of instruction-level rewriting methodologies that can insert jumps to 
trampolines while not changing the layout of the original program. Since this 
approach does not need a control-flow graph or any binary analysis, it can be 
easily applied to large-scale binaries.
%
\F~\ref{fig:e9patch} illustrates one trampoline example of \textsc{E9Patch}. 
The original instruction, \texttt{mov \%rax,(\%rbx)}, corresponds to the bytes 
sequence \texttt{48 03 48}, \textsc{E9Patch} only modify the first 3 bytes to 
change the instruction to a \texttt{jmpq} instruction. The target of this 
\texttt{jmpq} instruction is restricted to a small memory region (\texttt
{0x8348XXXX}), where the instrumented code could be placed. The \textsc
{E9Patch} approach is robust as no heuristics are used. However, it is still 
limited as not all x86 instructions could be leveraged with their techniques. 
According to the evaluation results, this approach could achieve around 99\% 
probability of successful instrumentation.

\begin{figure}[tb]
  \centering
  \includegraphics[width=0.6\textwidth]{fig/E9Patch.pdf}
  \caption{Illustration of \textsc{E9Patch} trampoline~\cite{duck2020binary}.}
  \label{fig:e9patch}
\end{figure}

% StochFuzz
In addition to all the static methods discussed above, one recent research 
proposed an interesting idea, \textsc{StochFuzz}, that solves the symbolization 
problem dynamically~\cite{zhang2021stochfuzz}. While admitting the difficulties 
that exist in control flow recovery and symbolization process, this approach 
does not try to design complicated analyses or heuristics. Instead, it 
accomplishes symbolization based on observations of runtime behavior, i.e., it 
dynamically verifies whether an address corresponds to data or code.

\begin{figure}[tb]
  \centering
  \includegraphics[width=1.0\textwidth]{fig/STOCHFUZZ.pdf}
  \caption{One example of \textsc{StochFuzz} rewriting strategies~\cite{zhang2021stochfuzz}.}
  \label{fig:stochfuzz}
\end{figure}

To be more specific, this approach proposes an incremental and stochastic 
rewriting technique that leverages the fuzzing technique to validate 
assumptions about symbols. It generates many different versions of rewritten 
binaries then tries to trigger specific behavior with fuzzing runs.
%
\F~\ref{fig:stochfuzz} illustrate one of the rewriting strategies used in 
\textsc{StochFuzz}. Given the program listed, the code will be first replace 
with \texttt{hlt} instruction which will case a segment fault, as shown in 
\textcircled{\raisebox{-0.9pt}{1}}. The segment faults, called \textit
{intentional crashes}, indicate that a code block that was not disassembled was 
found. \textsc{SotchFuzz} starts by fuzzing the rewritten binary. Once the 
segment fault occurs, the code block that starts from the current address will 
be disassembled and placed at a \textit{shadow space} marked as a afl 
trampoline in the figure. (Note that the instruction at address 110 is \texttt
{lea r8, [rip - 92]}, in which \texttt{rip - 92} corresponds to the original 
code address \texttt{rip + 8}.) After the disassembling, the new binary will be 
compiled and fuzzed again. Overall, this binary is recompiled with one \textit
{incremental} block of disassembled code each time. Therefore this process is 
called ``incremental rewriting''. Eventually, all covered blocks will be 
correctly disassembled by iteratively rewriting and fuzzing without ``data or 
code'' confusion.
