\chapter{Introduction}\label{sec-introduction}

\section{Motivation}

% Introduce the background knowledge 

Classification is the problem of identifying if an observation or object belongs to a set or not, or which of several sets. It is a fundamental problem in machine learning, data mining and computer vision. With the support of the increasing capacity of computation resources and growing volume of available data, the last decades have witnessed an explosion of breakthroughs in these fields. Nowadays, classification models (classifiers) are widely adopted to solve real world tasks, including face recognition [], handwritten recognition [], sentiment analysis [] and spam filtering []. Take image classification for instance, a well-designed convolutional neural network can achieve human-level performance in a number of benchmark datasets [,]. 

Despite their promising capability, an often-overlooked aspect is the important role of humans \cite{ribeiro2016kdd}. When humans are to understand and collaborate with these autonomous systems, it is desirable if we have explanations of their outputs. For instance, a doctor using a machine classifier to assist identifying early signs of lung cancers would need to know why the classifier ``thinks'' there might be a cancer so that he/she can make a more confident diagnosis. The research for explainable intelligent systems can be traced backed to 1980s, when expert systems are created and proliferated \cite{clancey1981tech, neches1985tse, swartout1991expert}. These early works focused on reducing the difficulty of maintaining the complicated if-then rules by designing more explainable representations. Recently, DARPA launches the Explainable Artificial Intelligence (XAI) project \cite{darpa2017xai}, which aims to develop new techniques to make the new generations of AI systems explainable.

However, the best-performing classifiers (e.g., neural networks) are becoming increasingly complex, in terms of the number of parameters and operations they employed, which makes them difficult to be explained. 

% Motivation of the topic, Importance

% Potential Impact

\section{Challenges and Research Issues}

% Summarize the research challenges of the topic

\section{Overview}

% An overview of the survey

\newpage
