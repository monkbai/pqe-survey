\chapter{Background}\label{sec-background}

Before the literature review of the existing research efforts, we first present
the preliminary knowledge of software reverse engineering, including the
domain-specific concepts and notations. In 1990, the definition of software
reverse engineering (SRE) is proposed by Institute of Electrical and
Electronics Engineers (IEEE) as ``the process of analyzing a subject system to
identify the system's components and their interrelationships and to create
representations of the system in another form or at a higher level of
abstraction'' in which the ``subject system'' is the end product of software
development~\cite{ieee-sre}.
For the last decades, as the basis of lots of security-critical tasks, software
reverse engineering techniques have been continuously explored and developed by
reverse engineers and researchers. Nowadays, modern software reverse
engineering has formed a three-component pipeline that consists of disassembly,
lifting, and decompilation.


\section{Workflow} \label{sec:background-workflow}
We demonstrate the workflow of modern software reverse engineering in
\F~\ref{fig:workflow}. A complete software reverse engineering process starts
with a binary executable without available source code. First, the disassembler
will parse and disassemble the binary file into assembly code with function
boundaries recovered at the disassembly stage. Next, the disassembled assembly
code, as the input of the lifter, will be lifted to (customized) IR with
variables and types recovered depending on the characteristics of IR. Lastly,
at the decompilation stage, the program control flow will be recovered and
optimized in order to be translated into the pseudo-high-level programming
language. With some advanced techniques, the output of the disassembler even
could be symbolized to be able to reassemble into a functional binary
executable. In this case, the assembly or IR can also be used to support
downstream applications without further processing.

\section{Concepts} \label{sec:background-concepts}
In this section we detail the concepts and basics of common components in the
software reverse engineering process.

\subsection{Disassembly} \label{subsec:background-disassembly}
Conceptually, disassembly is the process of translating machine code into
assembly language, it is the inverse process of
assembly~\cite{schwarz2002disassembly,wang2015reassembleable,bauman2018superset}.
%The tool conduct disassembly is called a disassembler, whose output is often
%formatted for human readability rather than suitability for input to an
%assembler.
Machine code is bytes that are read and executed by a machine. While machine
code is unreadable by humans, disassembly converts machine code into a
human-understandable form by converting hexadecimal values into assembly
language represented as visible characters.
For instance, each unique hexadecimal sequence on the left corresponds to one
and only one assembly instruction, in this case, disassembly is simply about
mapping the sequence to the assembly code:

\vspace*{3pt}
\noindent\hspace*{36pt}\begin{minipage}{.40\linewidth}
\begin{lstlisting}
 ; machine code
 0x55
 0x89 0xe5
 0x53
 0x83 0xec 0x74
\end{lstlisting}
\end{minipage}\hspace*{24pt}
\begin{minipage}{.40\linewidth}
\begin{lstlisting}
 ; disassembly code
 push ebp
 mov  ebp, esp
 push ebx
 sub  esp, 74h
\end{lstlisting}
\end{minipage}

\noindent \textbf{Linear and Recursive Disassembly.} While translating a
hexadecimal value sequence into an assembly instruction is straightforward,
the problem is where to find the hex values to translate. Classical static
disassembly techniques can be summarized as two methods. The first method,
named \textit{linear disassembly}, translates the hex values to assembly
instruction sequentially from the start of the text segment. However, in
modern compilers, data and code may be interleaved for optimization reasons.
Thus the second method, named \textit{recursive disassembly}, is leveraged to
circumvent the interleaving problem. Recursive disassembly will translate
instructions following the control flow of the program, so no data will be
misinterpreted as code. Nonetheless, statically recovering control flow from
binary is generally hard when indirect jumps exist (e.g., \textit{call eax})
~\cite{shoshitaishvili2016sok}.
Thus in practice, this problem is usually solved by combining linear and
recursive disassembly as \textit{speculative disassembly}
~\cite{cifuentes2000uqbt}.

\noindent \textbf{Reassembleable Disassembly.} While traditional disassembly
aims to readability and expressiveness only, Wang et al. first proposed the
\textit{reassembleable disassembly} which aims to produce output that can be
reassembled back to a functional program~\cite{wang2015reassembleable}. By
fixing the symbolization problem, reassembleable disassembly can enable lots
of binary rewriting applications including software security hardening and
cross-architecture code reuse. More details of the reassembleable disassembly
technique and related work will be discussed in \S~\ref{sec:existing-symbolization}.

In the process of software reverse engineering, input executables will first be
fed into the disassembler, while the machine code is translated into assembly
instructions, the data sections within each executable will also be identified
for further usage. Existing research on disassembly has been shown to work
very well in practice and the proposed methods can smoothly disassemble
large-size binary executables
~\cite{balakrishnan2010wysinwyx,kruegel2004static,wang2015reassembleable}.

\subsection{Lifting} \label{subsec:background-lifting}

With a variety of available IRs, LLVM IR, as a \textit{strong-typed} IR
defined for LLVM compiler framework, is used by most reverse engineering
tools. With the support of many LLVM analysis and optimization passes, LLVM IR
can directly bridge reverse engineering tool development and compiler
framework, avoid reimplementing the wheels by reusing existing LLVM passes,
and show great potential and research value. Therefore, we focus on LLVM IR
(hereinafter referred to as IR) and related lifting techniques in this survey.

The assembly code output by the disassembler will be fed to the lifter as
input. The machine-specific detail is discarded during lifting, leading to a
more refined platform-independent representation of the program (IR). The key
procedure of lifting is translating each assembly instruction to a sequence of
semantic-equivalent IR statements:

\vspace*{3pt}
\noindent\hspace*{36pt}\begin{minipage}{.40\linewidth}
\begin{lstlisting}
 ; assembly code
 mov  dowrd ptr [rbp -4], edi
\end{lstlisting}
\end{minipage}\hspace*{24pt}
\begin{minipage}{.40\linewidth}
\begin{lstlisting}[language=llvm]
 ; lifted IR
 %1 = add i64 %RBP, -12
 %2 = load i32, i32* %EDI
 %3 = inttoptr i64 %1 to i32*
 store i32 %2, i32* %3
\end{lstlisting}
\end{minipage}

The lifted IR statements would faithfully emulate machine execution, including
CPU register-level computations, memory updates, and other side effects. A
sequence of IR statements, corresponding to one machine instruction, will be
usually represented as a \textit{translation rule} and hardcoded in lifters.
Hence, each machine instruction will be mapped to a translation rule and
translated into an IR sequence.

\noindent \textbf{Emulation-style vs. Succinct-style.} The LLVM IR is a
strong-typed IR, which means all variables and their types are explicitly
defined in LLVM IR and branches are dependent on predicate variables
decidedly. Therefore, given the lifted IR code, the central focus is to
recover variables, types, and high-level program control flows from low-level
IR code. To recover program variables, some tools already proposed and
implemented needed static analysis and inference
techniques~\cite{anand2013compiler,balakrishnan2010wysinwyx,balakrishnan2007divine,reps2008improved,elwazeer2013scalable}.
To recover variable types, constraint-based type inference systems are
typically formulated~\cite{lee2011tie,noonan2016polymorphic}. However,
recovering variables and types from stripped binary is challenging and
existing methods are either inefficient or inaccurate enough (we will detail
the challenges and existing method at \S~\ref{sec:challenges-types} and \S~\ref{sec:existing-recovery}).

To avoid facing type and variable recovery challenges directly, lifter
developers have made compromises by trying to recover variables and types but
do not guarantee the correctness, which is denoted as \textit{succinct-style},
or trying to emulate the assembly instruction at hardware level thus ensuring
correctness at the cost of efficiency, which is denoted as \textit{emulation-style}.
We will compare the two lifting styles in detail in \S~\ref{sec:existing-llvm-lifting}.

\subsection{Decompilation} \label{subsec:background-decompilation}

\begin{quote}
\textit{``... machine-specific detail is discarded, leading to the essence of
the program (source code), from which it is possible to divine a deeper truth
(understanding of the program).''}~\cite{van2007static}
\end{quote}

Although IR is already understandable, the IR is designed with only limited
and straightforward kinds of basic operations (load, store, arithmetic
operations, etc.) for better analysis and optimizing friendliness. Limiting
operations to a few simple and basic ones allow the compiler developer to
dispense with complex operations considerations in an optimization strategy.
However, overly simple instructions make it difficult for analysts to
summarize and understand the higher-level intent of the program. Thus
decompilation technique is used to further extract the essence of a program by
converting it into a more advanced programming language-like
form~\cite{van2007static,fokin2010reconstruction,engel2011enhanced,brumley2013native}.

While the high-level data-flow information, including variables and types, is
recovered at the lifting stage, the control flow information in IR still
retains a simple assembly-like structure:

\vspace*{3pt}
\noindent\hspace*{36pt}\begin{minipage}{.40\linewidth}
\begin{lstlisting}
 ; branch in assembly
  cmp	 [rbp - 12], 5
  jge	 label
\end{lstlisting}
\end{minipage}\hspace*{24pt}
\begin{minipage}{.40\linewidth}
\begin{lstlisting}[language=llvm]
 ; branch in LLVM IR
  %2 = icmp slt i32 %1, 5
  br i1 %2, label %3, label %4
\end{lstlisting}
\end{minipage}

To recover high-level control flow structures, modern decompilers implement a
set of structure templates and search to determine whether an IR code region
matches the predefined patterns:

\vspace*{3pt}
\noindent\hspace*{36pt}\begin{minipage}{.40\linewidth}
\begin{lstlisting}[language=llvm]
 ; LLVM IR
  ...
  br label %4
4:
  ...
  %6 = icmp slt i32 %5, 5
  br i1 %6, label %7, label %13
7:
  ...
  br label %4
13:
  ...
  ret i32 0
\end{lstlisting}
\end{minipage}\hspace*{24pt}
\begin{minipage}{.40\linewidth}
\begin{lstlisting}[language=C]
 ; decompiled code
  while (x < 5) {
    ...
  }
  ...
  return 0;
\end{lstlisting}
\end{minipage}

Some advanced techniques enable an iterative refinement to polish the
recovered structure. To date, techniques have been designed to guarantee the
structure recovery correctness and also to improve
readability~\cite{brumley2013native,yakdan2015no}. In addition, modern
decompilers usually design optimizations to polish the lifted IR code,
including dead code elimination and untiling~\cite{brumley2013native,cifuentes1994reverse,kvroustek2017retdec}.
Also, reverse engineering of C executable files might encounter ``chicken and
egg'' problems (e.g., data flow analysis relies on the precise output of
control flow analysis, and vice versa). Indeed, analysis and optimization
modules can be invoked back and forth for iterations; the output of one module
is used to promote the analysis of other modules~\cite{kvroustek2017retdec}.

% \section{Application Scenarios} \label{sec:background-applications}
% TODO

\section{Survey Scope} \label{sec:background-scope}
In this survey, we focus on the challenges of software reverse engineering and
existing methods. Since software reverse engineering has been studied for over
decades, voluminous studies on various problems in reverse engineering have
been published in many domains.
To make this survey intuitive and concise, we focus on landmark solutions to
various problems in the field of software reverse engineering and recent
research published in top-tier conference and journal of the following areas:

\begin{itemize}
  \item \textbf{Security}: IEEE Symposium on Security and Privacy(S\&P),
  USENIX Security Symposium, ACM Conference on Computer and Communications
  Security (CCS), The Network and Distributed System Security Symposium (NDSS)
  \item \textbf{Programming Language}: ACM SIGPLAN Conference on Programming
  Language Design And Implementation (PLDI), Conference on Object-Oriented
  Programming Systems, Languages, and Applications (OOPSLA).
  \item \textbf{System}: ACM International Conference on Architectural Support
  for Programming Languages and Operating Systems (ASPLOS), USENIX Annul
  Technical Conference (USENIX ATC), ACM European Conference on Computer
  Systems (EuroSys).
  \item \textbf{Software Engineering}: International Conference on Software
  Engineering (ICSE),ACM Joint European Software Engineering Conference and
  Symposium on the Foundations of Software Engineering (ESEC/FSE), IEEE/ACM
  International Conference on Automated Software Engineering (ASE) and The ACM
  SIGSOFT International Symposium on Software Testing and Analysis (ISSTA).
\end{itemize}


\newpage
