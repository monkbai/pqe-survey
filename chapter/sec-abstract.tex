\begin{abstract}

Software reverse engineering is a practice of understanding the intentions of
software developers by analyzing and extracting design and implementation
information from compiled software. It works as an automatic process of
converting the incomprehensible binary files into human-readable high-level
language. As the cornerstone of various cybersecurity tasks, software reverse
engineering widely enables malware analysis, off-the-shelf software security
hardening, cross-architecture code reuse, and vulnerability detection, in which
cases the source code is usually unavailable.

Because of its importance, software reverse engineering has evolved over the
last few decades and has developed into a complete and systematic process.
Typically, software reverse engineering on the x86 platform consists of three
parts as disassembly, lifting, and decompilation. The disassembler translates
hex values in executable into assembly instructions at the disassembly stage.
Then the assembly instructions will be lifted into a variety of different
Intermediate Representations (IRs) by the lifter at the lifting stage. Finally,
at the decompilation stage, the decompiler converts IR into comprehensible
high-level language.

Accurate software reverse engineering is notoriously hard because the compiler
discards unnecessary information during converting source code to machine code.
Nowadays, there are lots of studies focused on recovering more information from
binary executables. However, the perfect solution remains do not exist.
Therefore, to help understand the challenges and the state-of-the-art (SOTA)
methodology of software reverse engineering techniques, we conduct a thorough
survey related to the existing researches and a detailed comparison among the
different methods at different stages.

According to the procedure of software reverse engineering, we first introduce
the difficulties in three stages: disassembly, lifting, and decompilation.
Afterward, we discuss recent works related to disassembly and reassembleable
disassembly techniques and their pros and cons. Then we compare and divide
existing lifting techniques into two categories of emulation-style and
succinct-style. Moreover, we also show the advanced decompilation techniques
for converting IR to pseudo C code. Finally, we discuss the emerging fields and
application scenarios of decompilation technology. After a thorough survey of
the existing literature, we also suggest several potential future research
directions that could drive the development of software reverse engineering
techniques. We believe our survey will benefit the entire reverse engineering
community.

\end{abstract}
