\begin{abstract}

Classification is a fundamental problem in machine learning, data mining, and computer vision. In practice, interpretability is a desirable property of classification models (classifiers) in critical areas, such as security, medicine, and finance. For instance, a quantitative trader may prefer a more interpretable model with less expected return due to its predictability and low risk. Unfortunately, the best-performing classifiers in many applications (e.g., deep neural networks) are complex models whose predictions are difficult to explain. Thus, there is a growing interest in using visualization to understand, diagnose and explain intelligent systems in both academia and industry. Many challenges need to be addressed in the formalization of explainability, and the design principles and evaluation of explainable intelligent systems. 

The survey starts with an introduction to the concept and background of explainable classifiers. Efforts towards more explainable classifiers are categorized into two types: designing classifiers with simpler structures that can be easily understood; developing methods that generate explanations for already complicated classifiers. Based on the life circle of a classifier, we discuss the pioneering work of using visualization to improve its explainability at different stages in the life circle.
%Existing work in both visualization and machine learning communities is categorized in terms of data type and the purpose of explanation. 
The survey ends with a discussion about the challenges and future research opportunities of explainable classifiers.

\end{abstract}
